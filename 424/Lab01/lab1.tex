\documentclass[12pt]{article}

\usepackage{amsmath}
\usepackage{setspace} 
\usepackage{graphicx} 
\usepackage{graphics} 
\usepackage{wrapfig} 
\usepackage{verbatim}
\singlespacing

\renewcommand\thesubsection{\thesection.\alph{subsection}}
\newenvironment{mylisting}
{\begin{list}{}{\setlength{\leftmargin}{1em}}\item\scriptsize\bfseries}
{\end{list}}


\begin{document}

\title{Music 424 Lab 1}
\author{Derek Tingle\\ derek.tingle@gmail.com}
\date{April 12, 2012}
\maketitle

% q1
\section{}
\subsection{}
$\surd$

\subsection{}

\section{}

\section{}

\subsection{}

\subsection{}
Let's say the input signal changes from $\mu$ to $\nu$ at $n=0$. So when $n<0, \lambda(n) = \mu$. When $n\geq0$,
\begin{eqnarray*}
\lambda (n) &=& (a\lambda(n-1)^p + (1-a)\nu^p)^{1/p} \\
\Rightarrow \lambda (0) &=& (a\mu^p + (1-a)\nu^p)^{1/p} \\
\Rightarrow \lambda (1) &=& (a\lambda(0)^p + (1-a)\nu^p)^{1/p} \\
&=& (a[(a\mu^p + (1-a)\nu^p)^{1/p}]^p + (1-a)\nu^p)^{1/p} \\
&=& (a(a\mu^p + (1-a)\nu^p) + (1-a)\nu^p)^{1/p} \\
&=& (a^2\mu^p + a(1-a)\nu^p + (1-a)\nu^p)^{1/p} \\
&=& (a^2\mu^p + (a+1)(1-a)\nu^p)^{1/p} \\
\Rightarrow \lambda (2) &=& (a\lambda(1)^p + (1-a)\nu^p)^{1/p} \\
&=& (a[(a^2\mu^p + (a+1)(1-a)\nu^p)^{1/p} ]^p + (1-a)\nu^p)^{1/p} \\
&=& (a(a^2\mu^p + (a+1)(1-a)\nu^p) + (1-a)\nu^p)^{1/p} \\
&=& (a^3\mu^p + (a^2+a+1)(1-a)\nu^p)^{1/p} \\
\Rightarrow \lambda (n) &=& (a^{n+1}\mu^p + (a^n+a^{n-1}\ldots+a+1)(1-a)\nu^p)^{1/p} \\
&=& (a^{n+1}\mu^p + (\frac{1-a^{n+1}}{1-a})(1-a)\nu^p)^{1/p} \\
&=& (a^{n+1}\mu^p + (1-a^{n+1})\nu^p)^{1/p} \\
\end{eqnarray*}

And we're interested in the time $n$ s.t.
\begin{eqnarray*}
\mu + (1-1/e)(\nu-\mu)&=& (a^{n+1}\mu^p + (1-a^{n+1})\nu^p)^{1/p} \\
\Rightarrow a^{n+1}(\mu^p-\nu^p)+\nu^p &=&[\mu+(1-1/e)(\nu-\mu)]^p \\
\Rightarrow a^{n+1} &=& \frac{[\mu+(1-1/e)(\nu-\mu)]^p - \nu^p}{\mu^p-\nu^p}\\
\end{eqnarray*}

When the signal is increasing (attack), $\mu < \nu$, and for simplicity, $\mu=0, \nu=1$
\begin{eqnarray*}
a^{n+1} &=& \frac{[\mu+(1-1/e)(\nu-\mu)]^p - \nu^p}{\mu^p-\nu^p} \\
\Rightarrow a^{n+1} &=&-(1-1/e)^p + 1\\
\Rightarrow n+1 &=& \frac{\ln(1-(1-1/e)^p)}{\ln(a)}\\
& \approx & \frac{\ln(-(1-1/e)^p)}{\ln(a)} \\
\Rightarrow n &=&  \frac{\ln(-(1-1/e)^p)}{\ln(a)} -1
\end{eqnarray*}

$n=\frac{-p}{\ln a}-1$

\end{document}

